
\documentclass[a4paper,UKenglish,cleveref, autoref, thm-restate]{lipics-v2021}
%This is a template for producing LIPIcs articles. 
%See lipics-v2021-authors-guidelines.pdf for further information.
%for A4 paper format use option "a4paper", for US-letter use option "letterpaper"
%for british hyphenation rules use option "UKenglish", for american hyphenation rules use option "USenglish"
%for section-numbered lemmas etc., use "numberwithinsect"
%for enabling cleveref support, use "cleveref"
%for enabling autoref support, use "autoref"
%for anonymousing the authors (e.g. for double-blind review), add "anonymous"
%for enabling thm-restate support, use "thm-restate"
%for enabling a two-column layout for the author/affilation part (only applicable for > 6 authors), use "authorcolumns"
%for producing a PDF according the PDF/A standard, add "pdfa"

%\graphicspath{{./graphics/}}%helpful if your graphic files are in another directory

\bibliographystyle{plainurl}% the mandatory bibstyle

\title{The 'Pandora' Cluster Editing Solver}

%\titlerunning{Dummy short title} %optional, please use if title is longer than one line

\author{Sebastian Paarmann}{Technische Universität Hamburg, Germany}{sebastian.paarmann@tuhh.de}{}{}

\authorrunning{S. Paarmann} %mandatory. First: Use abbreviated first/middle names. Second (only in severe cases): Use first author plus 'et al.'

\Copyright{Sebastian Paarmann} %mandatory, please use full first names. LIPIcs license is "CC-BY";  http://creativecommons.org/licenses/by/3.0/

\begin{CCSXML}
<ccs2012>
   <concept>
       <concept_id>10010405</concept_id>
       <concept_desc>Applied computing</concept_desc>
       <concept_significance>500</concept_significance>
       </concept>
   <concept>
       <concept_id>10002950.10003624.10003633.10010917</concept_id>
       <concept_desc>Mathematics of computing~Graph algorithms</concept_desc>
       <concept_significance>500</concept_significance>
       </concept>
   <concept>
       <concept_id>10002950.10003705.10003707</concept_id>
       <concept_desc>Mathematics of computing~Solvers</concept_desc>
       <concept_significance>300</concept_significance>
       </concept>
 </ccs2012>
\end{CCSXML}

\ccsdesc[500]{Applied computing}
\ccsdesc[500]{Mathematics of computing~Graph algorithms}
\ccsdesc[300]{Mathematics of computing~Solvers}

\keywords{Cluster Editing, fixed-parameter algorithms, graph theory, PACE challenge} %mandatory; please add comma-separated list of keywords

\category{} %optional, e.g. invited paper

\relatedversion{} %optional, e.g. full version hosted on arXiv, HAL, or other repository/website
%\relatedversiondetails[linktext={opt. text shown instead of the URL}, cite=DBLP:books/mk/GrayR93]{Classification (e.g. Full Version, Extended Version, Previous Version}{URL to related version} %linktext and cite are optional

% TODO: Mention source code here
\supplement{Source code available on Github and on Zenodo}%optional, e.g. related research data, source code, ... hosted on a repository like Zenodo, figshare, GitHub, ...
\supplementdetails[cite=paarmann_sebastian_2021_4964394]{Software}{https://github.com/spaarmann/cluster-editing} %linktext, cite, and subcategory are optional

%\funding{(Optional) general funding statement \dots}%optional, to capture a funding statement, which applies to all authors. Please enter author specific funding statements as fifth argument of the \author macro.

\acknowledgements{I want to thank Prof. Dr. Matthias Mnich and Dr. Jens M. Schmidt who were my
advisors on the Bachelor thesis that resulted in this work.}%optional

\nolinenumbers %uncomment to disable line numbering

% TODO: Probably do this?
\hideLIPIcs  %uncomment to remove references to LIPIcs series (logo, DOI, ...), e.g. when preparing a pre-final version to be uploaded to arXiv or another public repository

% TODO: ???
%Editor-only macros:: begin (do not touch as author)%%%%%%%%%%%%%%%%%%%%%%%%%%%%%%%%%%
\EventEditors{John Q. Open and Joan R. Access}
\EventNoEds{2}
\EventLongTitle{42nd Conference on Very Important Topics (CVIT 2016)}
\EventShortTitle{CVIT 2016}
\EventAcronym{CVIT}
\EventYear{2016}
\EventDate{December 24--27, 2016}
\EventLocation{Little Whinging, United Kingdom}
\EventLogo{}
\SeriesVolume{42}
\ArticleNo{23}
%%%%%%%%%%%%%%%%%%%%%%%%%%%%%%%%%%%%%%%%%%%%%%%%%%%%%%

\usepackage{mathtools}

% This declares \abs and \norm commands
\DeclarePairedDelimiter\abs{\lvert}{\rvert}%
\DeclarePairedDelimiter\norm{\lVert}{\rVert}%

% Swap the definition of \abs* and \norm*, so that \abs
% and \norm resizes the size of the brackets, and the
% starred version does not.
\makeatletter
\let\oldabs\abs
\def\abs{\@ifstar{\oldabs}{\oldabs*}}
%
\let\oldnorm\norm
\def\norm{\@ifstar{\oldnorm}{\oldnorm*}}
\makeatother

\begin{document}

\maketitle

%mandatory: add short abstract of the document
\begin{abstract}
	'Pandora' is a solver for the \textsc{Cluster Editing} problem developed for the PACE Challenge
	2021. It is based on a fixed-parameter bounded search tree algorithm and multiple data reduction
	rules.
\end{abstract}

\section{Introduction}
\label{sec:introduction}

This document provides a brief description of the techniques employed in creating the 'Pandora'
solver for the \textsc{Cluster Editing} problem. It is part of the submission of the solver to the
PACE Challenge 2021. % TODO: Cite?
The solver was written as part of a Bachelor thesis in Computer Science, and a much more detailed
description of most of the solver is also available in form of the thesis itself, although some
improvements have been made to the solver after the thesis was finalized. % TODO Cite/reference

As a brief introduction, the \textsc{Cluster Editing} problem asks to transform an arbitrary
unweighted, undirected input graph into a \emph{cluster graph}. A cluster graph is a graph in which
every connected component forms a clique, i.e.\ it consists only of one or more disjoint cliques.
This transformation should be achieved by producing a set of edges to either insert or delete from
the graph, and when solving the problem exactly, the goal is to find a \emph{minimal} set of edits
that results in a cluster graph.

Formally, for an input graph $G = (V, E)$ the goal is to find an $F \subseteq \binom{V}{2}$ such
that $(V, E \Delta F)$ is a cluster graph and $\abs{F}$ is minimal.

%\enlargethispage{\baselineskip} TODO?

\section{Bounded Search Tree Algorithm}

Internally, the input problem instance is converted into an equivalent instance for the
\textsc{Weighted Cluster Editing} problem. Here, the graph is characterized by a weight (or
\emph{cost}) function $s\colon \binom{V}{2} \to \mathbb{Z}$, such that $uv \in E \Leftrightarrow
s(uv) > 0$. For an edge $uv \in E$, $s(uv)$ are the deletion costs of the edge, while for a pair $uv
\notin E$, $-s(uv)$ are the insertion costs. The cost of a cluster editing $F$ is then $\sum_{e \in
F} \abs{s(e)}$ and the goal to find a cluster editing with minimal cost.

The solver implements a \emph{fixed-parameter algorithm}, where the problem has an additional
parameter $k$ which limits the allowed cost of a solution. Thus, for an input $(G, k)$ it should
return a solution $F$ if a solution with cost less than or equal $k$ exists, or 'no solution'
otherwise. To solve the original optimization problem, the algorithm is simply executed repeatedly
with increasing $k$ to find the minimal $k$ for which a solution can be found.

The basic structure of the algorithm is a \emph{bounded search tree}. The search tree strategy
implemented was introduced by Böcker et.\ al \cite{DBLP:journals/tcs/BockerBBT09} and
briefly repeated here: We choose an edge $uv \in E$ to branch and take two branches: Forbidding $uv$
and \emph{merging} $uv$. To forbid $uv$ we reduce $k$ by $s(uv)$ and set $s(uv) = -\infty$, deleting
the edge from the graph. Merging $uv$ means removing both $u$ and $v$ from the graph and replacing
them by a new vertex $u'$, setting $s(u'w) = s(uw) + s(vw) \forall w \in V \setminus \{u, v\}$. For
$s(uw) \neq -s(vw)$, $k$ can be reduced by $\min\{\abs{s(uw)}, \abs{s(vw}\}$. For $s(uw) = s(vw)$,
the resulting edge $u'w$ has weight $0$ which requires some additional care; this is described in
more detail in the original paper by Böcker et al. After merging $u$ and $v$ are considered to have
a permanent edge between them.

At every branching step we choose the edge $uv \in E$ that minimizes the \emph{branching number} of
the corresponding branching step. The branching number can be calculated from the branching vector
$(a, b)$ where $a$ is the cost of deleting $uv$ and $b$ is the cost of merging $uv$. For a
description of how a branching number can be derived from the branching vector, see e.g.\
\cite{DBLP:books/sp/CyganFKLMPPS15}.
In practice we have pre-computed a polynomial function that approximates the branching number from
the branching vector to allow quick computation. This branching strategy results in a search tree of
size $O(2^k)$.

Before each branching step we also calculate a simple lower bound for the remaining cost. If that
lower bound exceeds the current parameter $k$, we can immediately return 'no solution' for this
branch of the search tree.

\section{Reduction Rules}

The solver also implements multiple \emph{reduction rules} which can reduce the size or complexity
of a problem instance by essentially proving that certain edits must be part of an optimal solution
(in general, or in the current part of the search tree).

The first reduction is based on the idea of \emph{critical cliques} in the unweighted input graph:

\begin{definition}
	A \emph{critical clique} $K$ is an induced clique in $G$ where all vertices of $K$ have the same
    set of neighbors outside of $K$, and $K$ is maximal under this property.
\end{definition}

Guo proved that an optimal cluster editing will never split a critical clique into multiple clusters
\cite{DBLP:journals/tcs/Guo09}. As a first step in the algorithm, for every critical clique $K$, we
can merge all the vertices of $K$ into a single vertex, transforming the unweighted input into an
equivalent weighted one that the remainder of the algorithm works on.

We also implement a set of parameter-independent reductions that are applied once at the start, and
then again at regular intervals on the intermediate instances in the search tree. These were
described by Böcker et al.\ in \cite{DBLP:journals/algorithmica/BockerBK11}.
\subparagraph{Rule 1} \emph{(heavy non-edge rule)} Forbid a non-edge $uv$ with $s(uv) < 0$ if
\[
    \abs{s(uv)} \geq \sum_{w \in N(u)} s(uw).
\]

\subparagraph{Rule 2} \emph{(heavy edge rule, single end)} Merge vertices $u, v$ of an edge $uv$ if
\[
    s(uv) \geq \sum_{w \in V \setminus \{u, v\}} \abs{s(uw)}.
\]

\subparagraph{Rule 3} \emph{(heave edge rule, both ends)} Merge vertices $u, v$ of an edge $uv$ if
\[
    s(uv) \geq \sum_{w \in N(u) \setminus \{v\}} s(uw) + \sum_{w \in N(v) \setminus \{u\}} s(vw).
\]

\subparagraph{Rule 4} \emph{(almost clique rule)} Let $k_C$ be the min-cut value of $G[C]$, for $C
\subseteq V$. The vertices of $C$ can be merged if
\[
    k_C \geq \sum_{u,v \in C, s(uv) \leq 0} \abs{s(uv)}
        + \sum_{u \in C, v \in V \setminus C, s(uv) > 0} s(uv).
\]

\subparagraph{Rule 5} \emph{(similar neighborhood)} Merge $uv$ if
\begin{equation} \label{eq:rule5}
    s(uv) \geq \max_{C_u, C_v} \min\{s(v, C_v) - s(v, C_u) + \Delta_v, s(u, C_u) - s(u, C_v) +
    \Delta_u\}
\end{equation}
with the maximum running over all $C_u, C_v \subseteq W$ with $C_u \cap C_v = \emptyset$.

Böcker et al.\ \cite{DBLP:conf/apbc/BockerBBT08} also introduced a parameter-dependent reduction
that takes into account the current parameter $k$. We first define for each pair $uv$ the
\emph{induced costs} of forbidding or merging it:
\begin{align*}
    \mathsf{icf}(uv) &= \sum_{w \in N(u) \cap N(v)} \min\{s(uw), s(vw)\} \\
    \mathsf{icp}(uv) &= \sum_{w \in (N(u) \Delta N(v)) \setminus \{u, v\}}
        \min\{\abs{s(uw)}, \abs{s(vw)}\}
\end{align*}

One can then show that the following reduction rules are correct:
\begin{itemize}
	\item For $u, v \in V$ where $\mathsf{icf}(uv) + \max\{0, s(uv)\} + b(G, uv) > k$: Merge $u$ and $v$
	\item For $u, v \in V$ where $\mathsf{icp}(uv) + \max\{0, -s(uv)\} + b(G, uv) > k$: Forbid $uv$
\end{itemize}
where $b(G, uv)$ is a lower bound on the cost of an optimal solution for solving $G - \{u, v\}$.

And finally we also implemented a reduction rule derived from a kernelization based on edge cuts
introduced by Cao and Cen \cite{DBLP:journals/algorithmica/CaoC12}. The original kernelization and
reduction only works for graphs that do not contain any edges with weight $0$, however as we
mentioned before, such edges can be created by the merging procedure. We thus only apply the
reduction once at the beginning, before any zero edges can have been produced.

First, for a vertex $v \in V$ define the \emph{deficiency}
$\delta(v) = \sum_{x, y \in N(v), xy \notin E} -s(xy)$ and for a set $S \subseteq V$ define
$\gamma(S) = \sum_{x \in S, y \in V \setminus S, xy \in E} s(xy)$. Then the \emph{stable cost} of
$v$ is defined as $\rho(v) = 2\delta(v) + \gamma(N[v])$ and $N[v]$ is \emph{reducible} if $\rho(v) <
\abs{N[v]}$. If $N[v]$ is reducible, we can insert any missing edges in $G[N[v]$ to make it a
clique, reducing $k$ appropriately. Additionally, for each $x \in N(N[v])$, if the total weight of
edges between $x$ and $N[v]$ is less than or equal to the weight of all non-edges between $x$ and
$N[v]$, then we can delete all edges between $x$ and $N[v]$ and reduce $k$ appropriately.

%%
%% Bibliography
%%

%% Please use bibtex, 

\bibliography{pandora-solver-desc}

%\appendix

\end{document}
